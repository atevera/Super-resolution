\subsection{Modelado dinámico}
\begin{frame}{Diseño mecánico y asignación de referenciales}
        \begin{multicols}{2}
        \begin{figure}
            \begin{center}
                \includegraphics[scale=0.195]{Dedo Vist Isometrica.png}
                \caption{Modelo 3D}
            \end{center}
        \end{figure}
        
        \columnbreak
        
        \begin{figure}
            \begin{center}
                \includegraphics[scale=0.35]{ExoParametrizado.png}
                \caption{Referenciales}
            \end{center}
        \end{figure}
            
        \end{multicols}
\end{frame}

\begin{frame}{Parámetros estructura mecánica del exoesqueleto}
De acuerdo al modelo realizado en \emph{SolidWorks} y con base a
los códigos de movimiento en \cite{rigid_multibody}:
    
    \begin{table}[]
        \begin{figure}[htpb]
            \begin{center}
                \includegraphics[scale=0.3]{parametros_gryma.png}
                \label{fig:parametros_gryma}
            \end{center}
        \end{figure}
        \caption{Parámetros GRYMA}
    \end{table}
\end{frame}

\begin{frame}{Parámetros de actuadores}
    El \emph{Maxon DC-max 16 S} fue propuesto para cada articulación
    considerando las siguientes especificaciones:
    \begin{table}[H]
        \centering
        \begin{tabular}{cccc}
            Parámetro & Símbolo & Valor & Dimensión\\
            \hline \hline 
            Momento de Inercia & $J$ & $8e^{-9}$ & $[kgm^2]$\\ 
            Velocidad en vacío & $\dot{\theta}_N $ & $826.24$  & $[rad/s]$\\
            Corriente nominal & $I_0$ & $0.577$ & $[A]$\\
            Constante de par & $k$ & $0.00719$ & $[Nm/A]$ \\
            Relación mecánica & $N$ & $1$ &  \\
            Coeficiente de fricción viscosa & $b_e$ & $5.02e^{-6}$ &  $[Ns/m]$\\
        \end{tabular}
    \end{table}
    Considerando \cite{EcuacionFriccion}, 
    \begin{equation*}\label{Eqn:FricViscosa}
        b_e = \frac{kI_o}{\dot{\theta}_N} 
    \end{equation*}
\end{frame}

\begin{frame}{Cinemática de orden cero}
    Para cada referencial $\Sigma_i$ con articulaciones tipo revoluta:
    \begin{equation}
        A_i(q_i) = 
        \begin{bmatrix}
            e^{[\boldsymbol{\lambda}_{R_i}\times]q_i} & \boldsymbol{d}_{i}\\
            0 & 1
        \end{bmatrix}
        \label{eq:TH_GRYMA}
    \end{equation}
    
    Permitiendo la obtención de la cinemática directa para $\Sigma_i$:
    \begin{equation}
        A^i_0(\boldsymbol{q}) =  A^{p_i}_0(\boldsymbol{q}) A_i(q_i)
        \label{eq:CD}
    \end{equation}
\end{frame}

\begin{frame}{Cinemática de primer orden (1/2)}
    Para el Jacobiano Geométrico de Velocidad Angular:
    
    \begin{equation}
        ^0J_{\omega_i} =
        \left[\boldsymbol{\lambda}^{(0)}_{R_1}, \boldsymbol{\lambda}^{(0)}_{R_2}, \ldots, \boldsymbol{\lambda}^{(0)}_{R_i}, 0,\ldots, 0\right]
        \in \mathbb{R}^{3\times n}
        \label{eq:Jw}
    \end{equation}
    
    donde 
    \begin{equation}
        \boldsymbol{\lambda}^{(0)}_{R_i}(\boldsymbol{q}) = R^i_0(\boldsymbol{q})\boldsymbol{\lambda}_{R_i}
    \end{equation}
    
    Mientras que para el Jacobiano Geométrico de Velocidad Lineal: 
    \begin{equation}
        \left[^0J_{v_i}\right]_k =
        \begin{cases}
            \boldsymbol{\lambda}^{(0)}_{R_k} \times (\boldsymbol{d}_i - \boldsymbol{d}_k) & \text{si } k \leq i \\
            0                                                          & \text{si } k > 0
        \end{cases}
    \end{equation}
\end{frame}

\begin{frame}
    \frametitle{Cinemática de primer orden (2/2)}
    De esta manera, los Jacobianos geométricos expresados en el referencial 
    inercial se definen:

    \begin{equation}
        {^0}{}J_i(\boldsymbol{q}) = 
        \begin{bmatrix}
            {^0}{}J_{v_i}(\boldsymbol{q}) \\
            {^0}{}J_{\omega_i}(\boldsymbol{q})
        \end{bmatrix}
        \in \mathbb{R}^{6\times n}
    \end{equation}

    Obteniendo su expresión para los referenciales locales de la siguiente 
    forma:

    \begin{equation}
        {^i}{}J_i(\boldsymbol{q}) =
        \begin{bmatrix}
            {R^i_0}^T(q)  & {^0}{}J_{v_i}(\boldsymbol{q}) \\
            {R^i_0}^T(q) & {^0}{}J_{\omega_i}(\boldsymbol{q})
        \end{bmatrix} 
        \in \mathbb{R}^{6\times n} 
    \end{equation}

\end{frame}

\begin{frame}{Formulación de D'Alembert-Lagrange}
    Considerando \cite{3DMotion}:
    \begin{equation} 
        \label{eqn:DL_Equation}
         \frac{d}{dt} \frac{\partial K}{\partial \boldsymbol{\dot{q}}} - \frac{\partial K}{\partial \boldsymbol{q}} = \boldsymbol{Q}
         \in \mathbb{R}^n
    \end{equation}
    donde
    \begin{equation}
        \label{eqn:kinetic_energy}
         K = \frac{1}{2} \boldsymbol{\dot{q}}^T H(\boldsymbol{q}) \boldsymbol{\dot{q}}
    \end{equation}
    
    \begin{equation}
        \label{eqn:fuerzas_generalizadas}
         \boldsymbol{Q} \triangleq \begin{bmatrix} Q_1 \\ \vdots \\ Q_n \end{bmatrix}
    \end{equation}
    
    Resolviendo (\ref{eqn:DL_Equation}) y considerando $\boldsymbol{\tau}_U$ y $\boldsymbol{\tau}_D$, se obtiene:
    \begin{equation}
        \label{eqn:DL_final}
        H(\boldsymbol{q}) \boldsymbol{\ddot{q}} + C(\boldsymbol{q}, \boldsymbol{\dot{q}}) \boldsymbol{\dot{q}} + D(\cdot)\boldsymbol{\dot{q}}
        + \boldsymbol{g}(\boldsymbol{q}) = \boldsymbol{\tau}
    \end{equation}
\end{frame}

\begin{frame}{Matriz de Inercia}

    Partiendo de la expresión \eqref{eqn:kinetic_energy},
    
    \begin{equation}
        \label{eqn:inertia_matrix}
        H(\boldsymbol{q}) = \sum_{i=1}^n \ \; ^iJ_i^T(\boldsymbol{q}) \; M_i \; ^iJ_i(\boldsymbol{q}) 
    \end{equation}

    donde

    \begin{equation}
        \label{eqn:matriz_const} M_i = 
        \begin{bmatrix}
            m_i I & -m_i[\boldsymbol{r}_c{}_i \times] \\
            m_i[\boldsymbol{r}_{c_i} \times] & I_{c_i} -m_i[\boldsymbol{r}_{c_i} \times]^2\\
        \end{bmatrix}\in \mathbb{R}^{6\times6}
    \end{equation}
\end{frame}

\begin{frame}{Matriz de Coriolis y de Disipación}
    Para la construcción de la matriz de Coriolis $C(\boldsymbol{q},\boldsymbol{\dot{q}})$, el elemento 
    $(k,j)$ se define como:

    \begin{equation}
        c_{kj} = \sum_{i=1}^n \frac{1}{2}\left\{\frac{\partial h_{kj}(\boldsymbol{q})}{\partial q_{i}}+\frac{\partial h_{ki}(\boldsymbol{q})}{\partial q_{j}}-\frac{\partial h_{ij}(\boldsymbol{q})}{\partial q_{k}} \right\}\dot{q_i}
    \end{equation}
    
    El elemento $(i,j)$ de la matriz de disipación $D(\cdot)$ se expresa como:
    \begin{equation}
        D(\cdot)_{ij} = 
        \begin{cases}
            d & i = j \\
            0 & i \neq j
        \end{cases}
    \end{equation}

    Donde $d$ corresponde al coeficiente de fricción viscosa de cada articulación. 
\end{frame}

\begin{frame}{Energía Potencial y Vector de Gravedad}
    Considerando $U$ para sistemas multipartículas de acuerdo al referencial inercial,
    \begin{equation}
        \label{eqn:energia_potencial}
         U_{h} = \sum_{i=1}^n U_{h_{i}} =  -\sum_{i=1}^n m_i \boldsymbol{d}_{cm_i}^T \boldsymbol{g}_0
    \end{equation}
    Agregando, $U_0 = \left. U(\boldsymbol{q}) \right|_{\boldsymbol{q}=0}$
    \begin{equation}
        \label{eqn:energia_potencial_offset}
         U = U_{h}+U_0 \: \: : \dot{U} = \dot{U}_{h} 
    \end{equation}
    
    Mientras que el vector de gravedad:
    \begin{equation}
        \label{eqn:vector_gravedad}
        \boldsymbol{g}(\boldsymbol{q}) = \frac{\partial U}{\partial \boldsymbol{q}}
    \end{equation}
    siendo 
    \begin{equation}
        \label{eqn:g_0}
         \boldsymbol{g}_0 = \begin{bmatrix} 0 & 0 & -9.81 \end{bmatrix}^T
    \end{equation}
\end{frame}

\begin{frame}{Modelo dinámico del robot con actuadores}
    Se considera la nueva ecuación dinámica:
    \begin{equation}
        H_e(\boldsymbol{q})\boldsymbol{\ddot{q}} + C(\boldsymbol{q}, \boldsymbol{\dot{q}})\boldsymbol{\dot{q}}  + D_e(\cdot)\boldsymbol{\dot{q}} + g(\boldsymbol{q}) = \boldsymbol{\tau}
    \end{equation}
    donde
    \begin{align}
        H_e(\boldsymbol{q}) = H(\boldsymbol{q}) + J N^2 \\
        D_e(\cdot) = B N^2 + D(\cdot)
    \end{align}
    siendo el elemento $(i,j)$ de la matriz de disipación de los actuadores 
    $B$ se expresa como:

    \begin{equation}
        B_{ij} = 
        \begin{cases}
            b_e & i = j \\
            0 & i \neq j
        \end{cases}
    \end{equation}
    Donde $b_e$ corresponde al coeficiente de fricción de cada motor.

\end{frame}

\subsection{Control}
\begin{frame}{Linealización del sistema}
    Considerando la linealización del sistema por retroalimentación dada en \cite{spong_hutchinson_vidyasagar_2020},
    \begin{align}
        \boldsymbol{\dot{\overline{x}}} = A \, \boldsymbol{\overline{x}} + B \, \boldsymbol{\overline{u}} \;\;\; : \; \boldsymbol{x} -\boldsymbol{x}_0 \approx 0
    \end{align}
    siendo $\boldsymbol{x} = \begin{bmatrix} \boldsymbol{q} &  \boldsymbol{\dot{q}} \end{bmatrix}^T$. Entonces, 
    \begin{align}
        \overline{h}_{i} \ddot{q}_{i} + d_{e_i} \dot{q}_i = \tau_i 
        \label{eqn:modelo_joint}
    \end{align}
    O bien, 
    \begin{align}
        \label{eqn:joint_tf}
        G_j = \frac{1}{\overline{h}_{i} s^2 + B_i s} \;\;\;\; \to \;\;\;\;
        G_{2}(s) = \frac{\omega_n^2}{s^2 + 2 \xi \omega_n s + \omega_n^2}
    \end{align}
\end{frame}

\begin{frame}{Leyes de control (1/2)}
    Considerando las ecuaciones características, 
    \begin{block}{PD}
            \begin{itemize}
                \item $\boldsymbol{\tau}= k_p\boldsymbol{e}(t) + k_d\boldsymbol{\dot{e}}(t)$
            \end{itemize}
    \end{block}
    donde 
    \begin{align}
        \begin{split}
            k_{p_i} &= h_{e_i} \omega_n^2 \\
            k_{d_i} &= h_{e_i} \left(2 \xi \omega_n\right) - d_{e_i}
        \end{split}
    \end{align}
    siendo $t_s(1\%) = \frac{6}{\omega_n}$ con el fin de alcanzar un comportamiento \textbf{aproximado} al de un sistema de segundo orden. 

    \begin{block}{PD + g}
        \begin{itemize}
            \item $\boldsymbol{\tau} = k_p\boldsymbol{e}(t) + k_d\boldsymbol{\dot{e}}(t) + \boldsymbol{g}(\boldsymbol{q}^d)$
            \item $\boldsymbol{\tau} = k_p\boldsymbol{e}(t) + k_d\boldsymbol{\dot{e}}(t) + \boldsymbol{g}(\boldsymbol{q})$
        \end{itemize}
    \end{block}
\end{frame}

\begin{frame}{Leyes de control (2/2)}
    \begin{block}{PID}
        \begin{itemize}
            \item $\boldsymbol{\tau} = k_p \boldsymbol{e}(t) + k_d\boldsymbol{\dot{e}}(t) + k_i \int \boldsymbol{e}(t) dt$
        \end{itemize}
    \end{block}
    Si agregamos un polo dominante en $\lambda=10\omega_n$,
    \begin{align}
        \begin{split}
            k_{p_i} &= h_{e_i} \left(\omega_n^2 + 2 \xi \omega_n \lambda\right) \\
            k_{d_i} &= h_{e_i} \left(\omega_n^2 + 2 \xi \omega_n \right)- d_{e_i} \\
            k_{i_i} &= h_{e_i} \omega_n^2 \lambda
        \end{split}
    \end{align}
    Considerando la expresión de \emph{trabajo} $W_{1-2}$ en sistemas \emph{Lagrangianos},
    \begin{align}
        W_{1-2} \triangleq \int_{1}^{2} \boldsymbol{Q}^T \: d \boldsymbol{q} = E_\tau
    \end{align} 

\end{frame}