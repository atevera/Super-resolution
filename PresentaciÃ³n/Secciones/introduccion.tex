
\begin{frame}
    \frametitle{Introducción}
    \begin{block}{Super Resolución}
        El gran reto de estimar una imagen de alta resolución, teniendo de base una imagen de baja calidad es a lo que se le conoce como 
        \emph{Super Resolución}. La Super-Resolución para una sola imagen (SISR) es un problema que ha sido estudiado desde antes del siglo XXI.
    \end{block}

\end{frame}

\begin{frame}
    \frametitle{Introducción}
    \begin{block}{Super Resolución}
        Para este proyecto, se implementarán 3 diferentes métodos para la obtención de Super Resolución, se consideran: El algoritmo de Freeman \cite{freeman},
        la Super resolución por Redes Neuronales Convolucionales \cite{SRCNN} y las Redes Generativas Adversarias. \cite{SRGAN}
        \end{block}

\end{frame}


\begin{frame}

    \frametitle{Objetivos del proyecto}
    \begin{block}{Objetivos}
        \begin{enumerate}
            \item Implementar los conocimientos teóricos de cada modelo para generar código que
             permita la implementación del modelo.
            \item Obtención de imágenes de super Resolución. 
            \item Realizar una comparativa entre los 3 modelos con la finalidad de obtener
            una perspectiva sobre las ventajas y desventajas de cada uno.
        \end{enumerate}
    \end{block}

\end{frame}
