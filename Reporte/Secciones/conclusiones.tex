\noindent
El desarrollo de los diferentes métodos considerando sus fundamentos teóricos
permitió comprender de una mejora manera los diferentes algoritmos y 
enfoques para realizar la tarea de \emph{Súper Resolución}. Ya sea desde
medios clásicos por \emph{Visión por Computadora} o bien utilizando 
\emph{Visión Artificial} mediante \emph{Redes Neuronales} como fue el 
caso de los últimos dos métodos. 

De este modo, el comparativo realizado en este documento fue éxitoso,
ya que presenta fundamentos para presentar las diferentes ventajas y 
desventajas. Ya sea desde el tiempo de ejecución y/o entrenamiento,
su generalidad en la mejora de la resolución o bien a partir de los 
detalles que es capaz de predecir sin tenerlos explicitamente en la 
imagen de entrada. Además, se realizó el entrenamiento de dos 
redes neuronales diferentes y el desarrollo de un paquete para Python
para el método propuesto por \cite{freeman}.

Es evidente que los métodos actuales proveen mejores resultados tal 
como se ha comentado. Sin embargo, el método \emph{clásico} permite 
explotar explicitamente técnicas y métodos de diferentes áreas.

En conclusión, el objetivo del proyecto fue alcanzado al proveer 
tres enfoques diferentes para resolver la tarea de \emph{Súper Resolución}
y darle una aplicación mediante un \emph{Zoom Digital} como se presentó en la
sección de resultados. 