% Introducción a los tres enfoques y enfatizar el trabajo a mano por Freeman. 
\noindent
Bajo un enfoque \emph{clásico}, existen tres formas de mejorar la resolución 
de una imagen:

\begin{itemize}
    \item Amplificación de detalles existentes
    \item Suma de multiples frames
    \item Único frame
\end{itemize}

Para el primero de ellos, se realiza una amplificación de las frecuencias altas
(donde se encuentran los detalles existentes de la imagen). Resulta bastante 
sencillo aunque como desventaja principal es que amplifica de igual manera el ruido
presente en la imagen. 
El segundo de los métodos considera un frame de alta resolución dado una sequencia
de frames de baja resolución los cuales agregan definición al frame principal. Por
otro lado, el tercer método permite aproximar la información de alta calidad que 
no se encuentra en el frame o imagen original y que evidentemente no puede obtenerse
sólo amplificando las frecuencias altas. 