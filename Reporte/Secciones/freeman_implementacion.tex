\noindent
Dado que el objetivo del proyecto es comparar tres diferentes métodos 
de súper resolución, en esta sección se presentarán las actividades
requeridas para la implementación de los métodos. Considerando desde los 
requisitos de software, hardware y datos de entrenamiento en caso sean 
necesarios. 

\subsection{Example-Based Super-Resolution}
\noindent
Para el primero de ellos, se retoma la literatura expuesta en \cite{freeman}
comenzando con la generación del diccionario donde se encuentran relacionados 
los parches de baja y alta resolución. Previo a esto, es necesario pre-procesar
el conjunto de imágenes para su posterior segmentación. 

De acuerdo a las indicaciones, se deben tener pares de imágenes en alta y baja 
resolución. En particular, se ha considerado las primeras 13 imágenes del 
\emph{dataset} disponible en \cite{MIRFLICKR}.

Realizando un método iterativo que realiza un barrido bidimensional en cada imagen
resulta bastante sencillo obtener los parches correspondientes de alta y baja resolución. 
