% Descripción de lo que es una red convolucional
\subsection{Redes Convolucionales}
\noindent
Las Redes neuronales convolucionales son  un tipo de redes neuronales artificiales  donde las \emph{neuronas}
corresponden a campos receptivos de una manera muy similar a las neuronas en la corteza visual primaria (V1) de un cerebro
biológico.  Este tipo de red es una variación de un perceptrón multicapa, sin embargo, debido a que su aplicación es realizada
en matrices bidimensionales, son muy efectivas para tareas de visión artificial, como en la clasificación y segmentación 
de imágenes, entre otras aplicaciones.

\subsection{SRCNN}
Como se menciona en \cite{freeman},