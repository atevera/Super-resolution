\subsection{SRGAN}

El SISR (Single Image Super Resolution) es un problema inverso, \emph{it est}, que para una imagen de baja resolución puede haber muchas
imágenes diferentes de alta resolución que le correspondan, esto basado en la interpretación del metodo utilizado, ya que
el principio basico es añadir información para obtener imágenes de alta resolución.
Las CNN presentan un gran avance en la reconstrucción de imágenes de baja resolución a alta resolución,
sin embargo, debido al escalado de la imagen o el hecho de que la imagen que se busca mejorar presenta grandes
variaciones con respecto a las del dataset (\emph{Data Augmentation}) los resultados podrían ser no satisfactorios.


Una alternativa quelas GAN´s (Generative Adversarial Networks) cuyo funcionamiento está basado
en la estimación de modelos generadores, como mencionan Goodfellow et al. \cite{GANs}, esto es 
posible gracias al entrenamiento simultáneo de dos modelos, uno \emph{generador (G)} que obtiene 
la distribución de la entrada para generar datos falsos y el otro \emph{discriminador (D)} el cual se encarga de estimar 
la probabilidad de que la muestra provenga del dataset de entrenamiento y discernir así entre estos datos y 
los del modelo \emph{ generador (G)}.

El término \emph{antagónicas} como se menciona en \cite{SRGAN_Tesis}, se refiere a la dinámica 
competitiva que se mantiene entre los dos modelos. Por un lado,
el generador tiene por objetivo crear nuevos datos que sean indistinguibles del
conjunto de entrenamiento, mientras que el discriminador debe poder ser capaz
de distinguir cuáles son los datos creados y los reales, siendo los últimos los que corresponden
 al conjunto de entrenamiento. Esto resulta en un proceso iterativo donde estos dos modelos
 se desafían uno a otro, logrando un ajuste de parámetros
 que logran producir datos que se parezcan con gran acierto a los reales.
 

\begin{figure}[H]
    \begin{center}
      \includegraphics[scale = 0.9]{modelo_gen_disc.png}
      \caption{Modelo Generador y Discriminador}
      \label{Alexis1}
    \end{center}
\end{figure}

Profundizando un poco más en los componentes del algoritmo, el discriminador es una red neuronal convolucional que consta de muchas 
capas ocultas y una capa de salida, la principal diferencia aquí es que la capa de salida de las GAN puede tener solo dos salidas, 
a diferencia de las CNN, que pueden tener un número diferente de salidas con respecto a la cantidad de etiquetas en las que está entrenado.
La salida del discriminador puede ser 1 o 0 dependiendo de la función de activación que se aplique. Si la salida es 1, 
entonces los datos proporcionados son reales y si la salida es 0, entonces se refiere a ellos como datos falsos.

El discriminador está capacitado con los datos del dataset, con estos aprende a reconocer cómo se ven y qué características deben 
clasificarse como reales.




\begin{figure}[H]
    \begin{center}
      \includegraphics[scale = 0.3]{Discriminador.jpg}
      \caption{Discriminador}
      \label{Alexis2}
    \end{center}
\end{figure}


Por el contrario, el generador es una red neuronal convolucional inversa, hace exactamente lo opuesto de lo que hace una CNN, ya que 
a estas se les da una imagen real como entrada y se espera una etiqueta clasificada como salida, 
pero en el generador, un vector de valores aleatorios se da como señal de entrada 
y se espera una imagen falsa como salida, esta imagen debera aproximarse a la real dependiendo del criterio
del generador.



\begin{figure}[H]
    \begin{center}
      \includegraphics[scale = 0.3]{generador.png}
      \caption{Generador}
      \label{Alexis3}
    \end{center}
\end{figure}
    
    \subsubsection{Funciones de perdida.}






    
    \subsubsection{Función de activación.}


