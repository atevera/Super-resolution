\\ Algunos comentarios sobre el uso de LaTeX

IMPORTANTE: 
Antes de hacer cambios en los archivos en común como: main, paquetes o bibliografía,
favor de actualizar su copia del repositorio (pull) por si alguien ha hecho cambios.
Posteriormente hacer los cambios necesarios y finalmente hacer la actualización
al repositorio de Github (push). Esto para no tener problemas en el control 
de la versión. Para el resto de archivos, digamos que individuales, no es necesario 
hacerlo ya que sólo uno de nosotros realizará cambios en ellos. 

Para importar imágenes pueden copiar el código de abajo y sustituir
los campos según lo necesiten. Para este "ejemplo" pueden citar la 
figura con el comando \ref{fig:flores}. 

En realidad el comando \ref{} sirve para referenciar lo que sea: 
ecuaciones (hay uno mejor para esto), figuras, tablas, secciones, etc. 
Todo lo que le ponga una "etiqueta" con el comando \label{}. Por lo mismo,
a modo de filtro es que tiene el prefijo "fig" para identificar que 
se trata la etiqueta por ejemplo de una figura. 

\begin{figure}[H]
    \includegraphics[scale = \scaleFigures]{ Nombre de la figura }
    \centering
    \caption{ Título }
    \label{fig:flores}
\end{figure}

Para las ecuaciones, pueden usarse estas dos estructuras {equation} o {align}.
{Align} es más versátil y más funcional en mi opinión cuando se combina con 
algunas estructuras matemáticas. Sin embargo {equation} también hace la chamba. 

\begin{equation}
    \label{eqn:x1}
    x^2 = \frac{1}{2}
\end{equation}

\begin{align}
    \label{eqn:x2}
    x^2 = \frac{1}{2}
\end{align}

También está la estructura {aligned} pero sirve para estructuras en específico,
creo, creoooo... que para hacer ecuaciones en multilínea. 

Para citar la ecuación, sin importar la estructura que usen, les recomiendo que lo 
hagan con \eqref{eqn:x1} porque con \ref{} estará sin paréntesis y pues no está 
cool. De igual forma, las etiquetas tienen el prefijo "eqn" para identificar 
que se tratan de ecuaciones. 

-> Para todas las etiquetas, favor de usar títulos dentro de lo posible autodescriptibles. 


Para tablas, les recomiendo este sitio: https://www.tablesgenerator.com/

Ahí ustedes tendrán una interfaz gráfica para poder manipular datos y copiar
directamente el código generado ya a su archivo, bastante útil y rápido. 

Como ejemplo:

\begin{table}[]
    \begin{tabular}{|c|c|c|}
    \hline
    1  & Carga                  & \textgreater 100 kgf              \\ \hline
    2  & Carrera (Stroke)       & aprox. 100 cm                     \\ \hline
    3  & Tipo de tracción       & Cable dual, sencillo o con cadena \\ \hline
    4  & Montaje                & Trolley                           \\ \hline
    5  & Botonera de control    & Recta o Tipo teléfono             \\ \hline
    6  & Montaje                & Manual                            \\ \hline
    7  & Peso de balancín       & 20 - 30 kg                        \\ \hline
    8  & Tipo de alimentación   & Eléctrica / Neumática             \\ \hline
    9  & Alimentación eléctrica & 110 V                             \\ \hline
    10 & Presión de trabajo     & 6 - 8 Bar                         \\ \hline
    \end{tabular}
\end{table}


Agregar únicamente la letra "H" entre los corchetes y la etiqueta con \label{tb:tabla1} como ejemplo justo
después de iniciar la estructura "table" (entre las líneas 53 y 54).


Para citar alguna referencia pueden utilizar el comando \cite{}, por ejemplo como: 
\cite{3DMotion} (del súper libroo). Antes de hacer esto, agregar la referencia en 
el archivo bibliografia.tex y actualizar el repo. 

La bibliografía está en BibTeX, así que pueden utilizar cualquier generador de
formato o bien copiarlo de Google Scholar si lo están usando. 
Como generadores de BibTeX les recomiendo ampliamente Zotero (incluso para la tesis, spoiler).

Zotero: https://zbib.org/
Ahí pueden elegir el formato de la bibliografía; para este caso será BibTeX generic citation style. 

En fin, creo que es todo. Igual cualquier duda que tengan me comentan porfa con 
toda la confianza. Si no le sé, le buscamos pero de que queda, queda :)

Bai