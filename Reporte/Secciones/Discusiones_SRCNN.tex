\subsection{SRCNN}
Durante el desarrollo del módelo SRCNN se observó que, cuando se realiza el entrenamiento del módelo para un factor de escalamiento
de \textbf{4}, ya sea en el espacio de color $RGB$ ó $YC_rC_b$ aparentemente se tiene una asintota en $0.003$ de la métrica $MSE$,
y se requeriría realizar un re-entrenamiento de estos modelos para verificar o descartar la existencia de esta asintota. Para
nuestro caso, decidimos detener el entrenamiento de la red en este punto, ya que observamos que la minimización de nuestra función
de coste ($MSE$) era mínima. Lo mismo ocurre para los modelos SRCNN entrenados con un factor de escalamiento de 2 (En ambos
espacios de color), pero en estos módelos, debido a que la imagen de entrenamiento en baja resolución no ha "perdido" tantos
detalles respecto de la imagen original, es que su función de coste $MSE$ se logra reducir en mayor medida, observandose un valor
aparentemente asintótico en $0.0012$, aunque de igual forma que para los modelos con factor de escalamiento 4, habría que re-entrenar
el módelo para verificar esto. Cabe destacar que en este punto para mejorar los módelos es necesario ir ajustando la tasa de aprendizaje
$\alpha$ del optimizador \textbf{Adam} de forma adecuada.\\
Para el módelo entrenado en el espacio de color $YC_rC_b$ se utiliza únicamente el canal $Y$ para el entrenamiento de la red,
esto por que es en este canal en el que se \emph{concentra} la mayor parte de la información de la imagén, ya que si observamos
unicamente este canal por separado el resultado sería la imagen original pero en escala de grises, es por ello que resulta conveniente
este módelo, dado que entrenar la red unicamente sobre un canal hace que la cantidad de parámetros de la red se reduzca, lo cual
puede reducir el tiempo de entrenamiento de la red.