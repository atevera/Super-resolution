\documentclass[journal, onecolumn]{IEEEtran}
\IEEEoverridecommandlockouts

\usepackage[spanish]{babel}
\usepackage[utf8]{inputenc}
\usepackage{cite}
\usepackage{amsmath,amssymb,amsfonts}
\usepackage{algorithmic}
\usepackage{graphicx}
\usepackage{textcomp}
\usepackage{xcolor}
\usepackage{fancyhdr}
\usepackage{listings}
\usepackage{float}
\usepackage{blindtext}
\usepackage{newtxmath}
\usepackage{wrapfig}
\usepackage{mathtools}
\usepackage{breqn}
\usepackage{titlesec}
\usepackage{multirow}

\usepackage[flushleft]{threeparttable}
\usepackage{makecell,booktabs}

\usepackage[hyphens]{url}
\usepackage[hidelinks]{hyperref}


\lstdefinestyle{code}{%
backgroundcolor=\color{gray!1},
basicstyle=\ttfamily\small,
commentstyle=\color{green!60!black},
keywordstyle=\color{magenta},
stringstyle=\color{blue!50!red},
showstringspaces=false,
%captionpos=b,
numbers=left,
numberstyle=\footnotesize\color{gray},
numbersep=5pt,
%stepnumber=2,
tabsize=2,
%frame=L,
%framerule=1pt,
%rulecolor=\color{red},
breaklines=true,
}

\hypersetup{breaklinks=true}


\titlespacing*{\section}
{0pt}{2.0ex plus 1ex minus .2ex}{1.0ex plus .2ex}
\titlespacing*{\subsection}
{0pt}{2.0ex plus 1ex minus .2ex}{1.0ex plus .2ex}
\titlespacing*{\subsubsection}
{0pt}{2.0ex plus 1ex minus .2ex}{1.0ex plus .2ex}

\renewcommand\IEEEkeywordsname{Palabras Clave}


\def\BibTeX{{\rm B\kern-.05em{\sc i\kern-.025em b}\kern-.08em
    T\kern-.1667em\lower.7ex\hbox{E}\kern-.125emX}}


\graphicspath{ {../Imagenes/} }

\begin{document}
    \bstctlcite{IEEEexample:BSTcontrol}
    \title{Súper Resolución\\
    \small{Comparativa de técnicas}}

    
    \author{\IEEEauthorblockN{Madrigal-Custodio, Jesús A., Tevera-Ruiz Alejandro, Torres-Martínez Luis Á.\\}
    \IEEEauthorblockA{\textit{Departamento: Robótica y Manufactura Avanzada} \\
    \textit{Centro de Investigación y de Estudios Avanzados del Instituto Politécnico Nacional}}
    }

    \maketitle

    \begin{abstract}
        En el presente documento se explican los fundamentos, metodología y proceso de implementación 
        para el desarrollo de algoritmos de súper resolución bajo diversos enfoques con el objetivo... 
    \end{abstract}

    \begin{IEEEkeywords}
    Súper Resolución, Redes Convolucionales, Inteligencia Artificial
    \end{IEEEkeywords}

    \section{Introducción}
    

    \section{Antecedentes}



    \section{Implementación}

    \section{Resultados}

    \section{Discusión}


    \section{Conclusiones}

    \nocite{*}

    \Urlmuskip=0mu plus 1mu\relax
    \bibliographystyle{IEEEtran}
    \bibliography{bibliografia}
    
    \section{Anexos}


\end{document}