\documentclass[journal, onecolumn]{IEEEtran}
\IEEEoverridecommandlockouts


% En caso de requerir agregar algún paquete, favor de ponerlo en el archivo
% "paquetes.tex". Ahí ya están algunos, los más comunes para importar imágenes, 
% ecuaciones, símbolos e importar código. Según yo son los necesarios para hacer
% el reporte. Pero de igual forma verificar y agregar según conveniencia. 
\usepackage[spanish]{babel}
\usepackage[utf8]{inputenc}
\usepackage{cite}
\usepackage{amsmath,amssymb,amsfonts}
\usepackage{algorithmic}
\usepackage{graphicx}
\usepackage{textcomp}
\usepackage{xcolor}
\usepackage{fancyhdr}
\usepackage{listings}
\usepackage{float}
\usepackage{blindtext}
\usepackage{newtxmath}
\usepackage{wrapfig}
\usepackage{mathtools}
\usepackage{breqn}
\usepackage{titlesec}
\usepackage{multirow}

\usepackage[flushleft]{threeparttable}
\usepackage{makecell,booktabs}

\usepackage[hyphens]{url}
\usepackage[hidelinks]{hyperref}


\lstdefinestyle{code}{%
backgroundcolor=\color{gray!1},
basicstyle=\ttfamily\small,
commentstyle=\color{green!60!black},
keywordstyle=\color{magenta},
stringstyle=\color{blue!50!red},
showstringspaces=false,
%captionpos=b,
numbers=left,
numberstyle=\footnotesize\color{gray},
numbersep=5pt,
%stepnumber=2,
tabsize=2,
%frame=L,
%framerule=1pt,
%rulecolor=\color{red},
breaklines=true,
}

\hypersetup{breaklinks=true}


\titlespacing*{\section}
{0pt}{2.0ex plus 1ex minus .2ex}{1.0ex plus .2ex}
\titlespacing*{\subsection}
{0pt}{2.0ex plus 1ex minus .2ex}{1.0ex plus .2ex}
\titlespacing*{\subsubsection}
{0pt}{2.0ex plus 1ex minus .2ex}{1.0ex plus .2ex}

\renewcommand\IEEEkeywordsname{Palabras Clave}


\def\BibTeX{{\rm B\kern-.05em{\sc i\kern-.025em b}\kern-.08em
    T\kern-.1667em\lower.7ex\hbox{E}\kern-.125emX}}


\graphicspath{ {../Imagenes/} }

\numberwithin{equation}{section}
\numberwithin{figure}{section}
\numberwithin{table}{section}

\newcommand{\scaleFigures}{0.5}

\begin{document}
    \bstctlcite{IEEEexample:BSTcontrol}
    \title{Métodos para súper-resolución de imágenes}
    
    \author{\IEEEauthorblockN{Madrigal-Custodio Jesús A., Tevera-Ruiz Alejandro, Torres-Martínez Luis Á.\\}
    \IEEEauthorblockA{\textit{Departamento: Robótica y Manufactura Avanzada} \\
    \textit{Centro de Investigación y de Estudios Avanzados del Instituto Politécnico Nacional}}
    }

    \maketitle

    \begin{abstract}
        En el presente documento se explican los fundamentos, metodología y proceso de implementación 
        para el desarrollo de algoritmos de súper resolución bajo diversos enfoques con el objetivo... 
    \end{abstract}

    \begin{IEEEkeywords}
    Súper Resolución, Redes Convolucionales, Inteligencia Artificial
    \end{IEEEkeywords}

    \section{Introducción}
    % Comando para importar archivos en la plantilla.
    % Para ello deberá crear el archivo .tex en la carpeta "Secciones" 
    % Y posteriormente importarlo con el comando "input".

    % Cada que realice un cambio en el main favor de actualizarlo en el 
    % repo para no tener problemas en el manejo de este archivo que todos
    % moveremos. Lo mismo para el archivo paquetes.
    
    % Para el resto no es necesario. 


    
El gran reto de estimar una imagen de alta resolución, teniendo de base una imagen de baja calidad es a lo que se le conoce como 
\emph{Super Resolución},la Super-Resolución para una sola imagen (SISR) es un problema que ha sido estudiado desde antes del siglo XXI, lo que comenzó tiempo atrás
como un tema de ciencia ficción termino por ser un tema de investigación que aun no culmina, pero presenta avances significativos en la actualidad.
Existen muchos antecedentes sobre modelos que tratan de recuperar la información de las imágenes de baja resolución haciendo un escalado de esta y 
 con el uso de modelos probabilísticos estimar los datos faltantes, otros que aplican parches en combinación con imágenes de alta resolución para lograr un 
 resultado legible sentando un precedente en la investigación.
 
 A pesar de estos esfuerzos aun se buscaba mejorar estas estimaciones, con el surgimiento de las \emph{Redes Neuronales Artificiales}, capaces de realizar predicciones dada sus
habilidad de aprendizajes, podían obtener modelos precisos de una tarea especifica a partir de fragmentos de información con los cuales realizan un entrenamiento. De
esta nueva estrategia surgen entonces diferentes modelos de redes Neuronales como los son las Convolucionales, GAN´s, Residuales, entre otras.Esto 
se puede considerar hasta ahora el mayor avance en cuanto a super resolución.

El objetivo de este trabajo es realizar una comparativa entre 3 diferentes métodos para la obtención de Super Resolución, se consideran el algoritmo de Freeman \cite{freeman},
la Super resolución por Redes Neuronales Convolucionales \cite{SRCNN} y las Redes Generativas Adversarias \cite{SRGAN}. Al aplicar estos modelos se busca
visualizar claramente como se estructura cada uno, cuales son las ventajas y desventajas, cuantificar el avance tecnológico y dar un panorama sobre lo alcanzado en la actualidad
mediante la comparativa de resultados. 
    

    \section{Antecedentes}
    % Introducción a los tres enfoques y enfatizar el trabajo a mano por Freeman. 
\noindent
Bajo un enfoque \emph{clásico}, existen tres formas de mejorar la resolución 
de una imagen:

\begin{itemize}
    \item Amplificación de detalles existentes
    \item Suma de múltiples frames
    \item Único frame
\end{itemize}

Para el primero de ellos, se realiza una amplificación de las frecuencias altas
(donde se encuentran los detalles existentes de la imagen) dada la variación local
entre los pixeles vecinos. 
La amplificación de detalles existentes resulta bastante 
sencillo de aplicar. Sin embargo, ante imágenes con una cantidad considerable de
ruido puede no ser la mejor opción a tomar. Además, al potencializar las 
frecuencias ya existentes de la imagen, el resultado estará definido por el detalle
previo en la imagen de entrada. 

El segundo de los métodos considera que el frame de alta resolución es el resultado
de una secuencia de frames de baja resolución que permiten obtener las frecuencias 
altas de la imagen resultante para mejorar su resolución. Esto es conveniente cuando
ya se cuenta con el conjunto de imágenes requeridas y se planea realizar una 
reconstrucción de la imagen en una mejor calidad.

Por otro lado, el tercer método basado en un único frame o imagen busca aproximar las
frecuencias altas (detalles) que no se encuentra en la entrada del algoritmo y que evidentemente
no puede obtenerse sólo amplificando las frecuencias altas como lo que ocurre con 
el primero de los métodos. 

\subsubsection{Interpolación}
\noindent
Para mejorar la calidad se busca aumentar la densidad de pixeles de la imagen con el objetivo
de hacer la imagen más grande y mejorar sus detalles a partir de la predicción de 
pixeles que no se encuentran en la imagen visiblemente, pero que podrían aproximarse
al buscar que se mantenga una consistencia en la imagen modificada de acuerdo a la
vecindad de los pixeles. 

Esto permite proponer el uso de algoritmos de interpolación que buscan predecir los 
pixeles vecinos y con ello aumentar la densidad de pixeles de la imagen de entrada. 
Con base en \cite{interpolation_cambridge}, dichos algoritmos pueden agruparse en dos categorías: adaptativos y no adaptativos.
Los primeros cambian dependiendo de lo que se está interpolando (bordes o texturas 
suaves) pixel por pixel con el objetivo de minimizar los errores antiestéticos
de los algoritmos de interpolación como el desenfoque o pérdida de detalles en 
regiones evidentes. Ejemplos de ellos pueden ser los softwares de licencia como
\emph{Qimage, PhotoZoom Pro, Genuine Fractals, etc}. 

Mientras que los métodos no adaptativos tratan todos los pixeles por igual
dada la predicción de un pixel central de acuerdo a sus pixeles adyacentes. Esto 
involucra que entre más vecinos se consideren en la interpolación, una mejor 
aproximación se tendrá del pixel a predecir, pero de manera proporcional 
aumentarán los recursos computacionales necesarios. Dentro de los algoritmos 
se incluyen: \emph{vecino más cercano, bilineal, bicúbica, spline, entre otros.}

A continuación se describirán algunos de los algoritmos no adaptativos para 
interpolación que serán utilizados en los diferentes métodos para \emph{Súper
Resolución}:
\begin{itemize}
    \item \textbf{Vecino más cercano} - Dado un pixel considera sólo un pixel 
    adyacente para la interpolación, lo que resulta en un menor tiempo de procesamiento
    pero resultados poco consistentes al observar al conjunto de pixeles interpolados. 
    \item \textbf{Bilineal} - Considera una vecindad 2x2 correspondiente al pixel
    a predecir con su correspondiente promedio ponderado de acuerdo a la distancia
    del pixel desconocido. Esto da como resultado un aspecto más suave que el vecino
    más cercano. 
    \item \textbf{Bicúbica} - Valora una vecindad 4x4 de pixeles conocidos para la
    predicción del pixel central considerando el mismo procedimiento de la 
    interpolación bilineal. Como resultado, se alcanzan imágenes más nítidas que los 
    métodos anteriores. Logrando así un equilibrio entre la calidad de salida y el 
    tiempo de procesamiento. Lo anterior promueve que sea un estándar en muchos programas
    de edición de imágenes, controladores de impresoras e interpolación en cámaras. 
\end{itemize}


\textbf{NOTA: Agregar comparativo de interpolaciones con un parche y escalado x2}





    \section{Implementación}

    \section{Resultados}

    \section{Discusión}


    \section{Conclusiones}
    \noindent buenas buenas 


    \nocite{*}

    \Urlmuskip=0mu plus 1mu\relax
    \bibliographystyle{IEEEtran}
    \bibliography{bibliografia}

\end{document}