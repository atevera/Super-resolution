\documentclass[journal, onecolumn]{IEEEtran}
\IEEEoverridecommandlockouts


% En caso de requerir agregar algún paquete, favor de ponerlo en el archivo
% "paquetes.tex". Ahí ya están algunos, los más comunes para importar imágenes, 
% ecuaciones, símbolos e importar código. Según yo son los necesarios para hacer
% el reporte. Pero de igual forma verificar y agregar según conveniencia. 
\usepackage{hyperref}
\usepackage{ctex}
\usepackage[spanish]{babel}
\usepackage{multicol}
\usepackage{multirow}
\usepackage{YTU}
\usepackage{pgfpages}
\usepackage{latexsym,amsmath,xcolor,multicol,booktabs,calligra}
\usepackage{graphicx,listings,stackengine}
\usepackage{lmodern}
\usepackage{ragged2e}
\decimalpoint
\hypersetup{breaklinks=true}


\titlespacing*{\section}
{0pt}{2.0ex plus 1ex minus .2ex}{1.0ex plus .2ex}
\titlespacing*{\subsection}
{0pt}{2.0ex plus 1ex minus .2ex}{1.0ex plus .2ex}
\titlespacing*{\subsubsection}
{0pt}{2.0ex plus 1ex minus .2ex}{1.0ex plus .2ex}

\renewcommand\IEEEkeywordsname{Palabras Clave}


\def\BibTeX{{\rm B\kern-.05em{\sc i\kern-.025em b}\kern-.08em
    T\kern-.1667em\lower.7ex\hbox{E}\kern-.125emX}}


\graphicspath{ {../Imagenes/} }

\numberwithin{equation}{section}
\numberwithin{figure}{section}
\numberwithin{table}{section}

\newcommand{\scaleFigures}{0.5}
\renewcommand{\thesection}{\arabic{section}}
%\renewcommand{\thesubsection}{\arabic{subsection}}


\begin{document}
    \renewcommand{\tablename}{Tabla}
    \bstctlcite{IEEEexample:BSTcontrol}
    \title{Métodos para súper-resolución de imágenes}
    
    \author{\IEEEauthorblockN{Madrigal-Custodio Jesús A., Tevera-Ruiz Alejandro, Torres-Martínez Luis Á.\\}
    \IEEEauthorblockA{\textit{Departamento: Robótica y Manufactura Avanzada} \\
    \textit{Centro de Investigación y de Estudios Avanzados del Instituto Politécnico Nacional}}
    }

    \maketitle

    \begin{abstract}
        En el presente documento se explican los fundamentos, metodología y proceso de implementación 
        para el desarrollo de algoritmos de súper resolución bajo diversos enfoques con el objetivo... 
    \end{abstract}

    \begin{IEEEkeywords}
    Súper Resolución, Redes Convolucionales, Inteligencia Artificial
    \end{IEEEkeywords}

    \section{Introducción}
    % Comando para importar archivos en la plantilla.
    % Para ello deberá crear el archivo .tex en la carpeta "Secciones" 
    % Y posteriormente importarlo con el comando "input".

    % Cada que realice un cambio en el main favor de actualizarlo en el 
    % repo para no tener problemas en el manejo de este archivo que todos
    % moveremos. Lo mismo para el archivo paquetes.
    
    % Para el resto no es necesario. 


    
El gran reto de estimar una imagen de alta resolución, teniendo de base una imagen de baja calidad es a lo que se le conoce como 
\emph{Super Resolución},la Super-Resolución para una sola imagen (SISR) es un problema que ha sido estudiado desde antes del siglo XXI, lo que comenzó tiempo atrás
como un tema de ciencia ficción termino por ser un tema de investigación que aun no culmina, pero presenta avances significativos en la actualidad.
Existen muchos antecedentes sobre modelos que tratan de recuperar la información de las imágenes de baja resolución haciendo un escalado de esta y 
 con el uso de modelos probabilísticos estimar los datos faltantes, otros que aplican parches en combinación con imágenes de alta resolución para lograr un 
 resultado legible sentando un precedente en la investigación.
 
 A pesar de estos esfuerzos aun se buscaba mejorar estas estimaciones, con el surgimiento de las \emph{Redes Neuronales Artificiales}, capaces de realizar predicciones dada sus
habilidad de aprendizajes, podían obtener modelos precisos de una tarea especifica a partir de fragmentos de información con los cuales realizan un entrenamiento. De
esta nueva estrategia surgen entonces diferentes modelos de redes Neuronales como los son las Convolucionales, GAN´s, Residuales, entre otras.Esto 
se puede considerar hasta ahora el mayor avance en cuanto a super resolución.

El objetivo de este trabajo es realizar una comparativa entre 3 diferentes métodos para la obtención de Super Resolución, se consideran el algoritmo de Freeman \cite{freeman},
la Super resolución por Redes Neuronales Convolucionales \cite{SRCNN} y las Redes Generativas Adversarias \cite{SRGAN}. Al aplicar estos modelos se busca
visualizar claramente como se estructura cada uno, cuales son las ventajas y desventajas, cuantificar el avance tecnológico y dar un panorama sobre lo alcanzado en la actualidad
mediante la comparativa de resultados. 
    \subsection{SRGAN}

La GAN´s (Generative Adversarial Networks) son un tipo de redes, cuyo funcionamiento está basado
en la estimación de modelos generadores, esto es posible gracias al entrenamiento de un modelo 
\emph{generador G} que obtiene la distribución de los datos y un modelo \emph{discriminador D} 
el cual se encarga de discernir entre los datos provenientes del dataset de entrenamiento en vez
del modelo \emph{G}.

    \section{Antecedentes}
    % Introducción a los tres enfoques y enfatizar el trabajo a mano por Freeman. 
\noindent
Bajo un enfoque \emph{clásico}, existen tres formas de mejorar la resolución 
de una imagen:

\begin{itemize}
    \item Amplificación de detalles existentes
    \item Suma de múltiples frames
    \item Único frame
\end{itemize}

Para el primero de ellos, se realiza una amplificación de las frecuencias altas
(donde se encuentran los detalles existentes de la imagen) dada la variación local
entre los pixeles vecinos. 
La amplificación de detalles existentes resulta bastante 
sencillo de aplicar. Sin embargo, ante imágenes con una cantidad considerable de
ruido puede no ser la mejor opción a tomar. Además, al potencializar las 
frecuencias ya existentes de la imagen, el resultado estará definido por el detalle
previo en la imagen de entrada. 

El segundo de los métodos considera que el frame de alta resolución es el resultado
de una secuencia de frames de baja resolución que permiten obtener las frecuencias 
altas de la imagen resultante para mejorar su resolución. Esto es conveniente cuando
ya se cuenta con el conjunto de imágenes requeridas y se planea realizar una 
reconstrucción de la imagen en una mejor calidad.

Por otro lado, el tercer método basado en un único frame o imagen busca aproximar las
frecuencias altas (detalles) que no se encuentra en la entrada del algoritmo y que evidentemente
no puede obtenerse sólo amplificando las frecuencias altas como lo que ocurre con 
el primero de los métodos. 

\subsection{Interpolación}
\noindent
Para mejorar la calidad se busca aumentar la densidad de pixeles de la imagen con el objetivo
de hacer la imagen más grande y mejorar sus detalles a partir de la predicción de 
pixeles que no se encuentran en la imagen visiblemente, pero que podrían aproximarse
al buscar que se mantenga una consistencia en la imagen modificada de acuerdo a la
vecindad de los pixeles. 

Esto permite proponer el uso de algoritmos de interpolación que buscan predecir los  3
pixeles vecinos y con ello aumentar la densidad de pixeles de la imagen de entrada. 
Con base en \cite{interpolation_cambridge}, dichos algoritmos pueden agruparse en dos
categorías: adaptativos y no adaptativos. Los primeros cambian dependiendo de lo que
se está interpolando (bordes o texturas suaves) pixel por pixel con el objetivo de
minimizar los errores antiestéticos de los algoritmos de interpolación como el
desenfoque o pérdida de detalles en regiones evidentes. Ejemplos de ellos pueden
ser los softwares de licencia como \emph{Qimage, PhotoZoom Pro, Genuine Fractals, etc}. 

Mientras que los métodos no adaptativos tratan todos los pixeles por igual
dada la predicción de un pixel central de acuerdo a sus pixeles adyacentes. Esto 
involucra que entre más vecinos se consideren en la interpolación, una mejor 
aproximación se tendrá del pixel a predecir, pero de manera proporcional 
aumentarán los recursos computacionales necesarios. Dentro de los algoritmos 
se incluyen: \emph{vecino más cercano, bilineal, bicúbica, spline, entre otros.}

A continuación se describirán algunos de los algoritmos no adaptativos para 
interpolación que serán utilizados en los diferentes métodos para \emph{Súper
Resolución}:
\begin{itemize}
    \item \textbf{Vecino más cercano} - Dado un pixel considera sólo un pixel 
    adyacente para la interpolación, lo que resulta en un menor tiempo de procesamiento
    pero resultados poco consistentes al observar al conjunto de pixeles interpolados. 
    \item \textbf{Bilineal} - Considera una vecindad 2x2 correspondiente al pixel
    a predecir con su correspondiente promedio ponderado de acuerdo a la distancia
    del pixel desconocido. Esto da como resultado un aspecto más suave que el vecino
    más cercano. 
    \item \textbf{Bicúbica} - Valora una vecindad 4x4 de pixeles conocidos para la
    predicción del pixel central considerando el mismo procedimiento de la 
    interpolación bilineal. Como resultado, se alcanzan imágenes más nítidas que los 
    métodos anteriores. Logrando así un equilibrio entre la calidad de salida y el 
    tiempo de procesamiento. Lo anterior promueve que sea un estándar en muchos programas
    de edición de imágenes, controladores de impresoras e interpolación en cámaras. 
\end{itemize}

En la Figura \ref{fig:interpoladores} se presentan los tres algoritmos no adaptativos
más utilizados. Observe que las definiciones dadas anteriormente coinciden con los 
resultados expuestos donde el \emph{vecino más cercano} resulta poco útil para
predecir los pixeles intermedios mientras que \emph{bilineal} o \emph{bicúbica}
predicen de mejor manera los pixeles intermedios con un poco más de detalle para
en el caso del último algoritmo. 
 
\begin{figure}[H]
    \includegraphics[scale = 0.6 ]{ tipos_interpoladores.png }
    \centering
    \caption{ Algoritmos de interpolación no adaptativos }
    \label{fig:interpoladores}
\end{figure}
Todos los interpoladores no adaptativos intentan encontrar un equilibrio óptimo
entre tres efectos no deseados: halos de borde, desenfoque y \emph{aliasing}. En la 
Figura \ref{fig:efectos_inter} puede observarse el efecto para cada caso. 

\begin{figure}[H]
    \includegraphics[scale = 0.5 ]{ efectos_inter.png }
    \centering
    \caption{ Efectos de interpolación}
    \label{fig:efectos_inter}
\end{figure}

Incluso los interpoladores no adaptativos más avanzados siempre tienden a aumentar
o disminuir algunos de los efectos a expensas de los otros dos, por lo tanto uno 
será más evidente. 

En contraste, los interpoladores adaptativos pueden o no producir los efectos 
mencionados aunque generalmente inducen texturas que no son de la imagen o 
pixeles extraños a pequeña escala.


    \subsubsection{Example Based Super Resolution}
mucho texto
    % Descripción de lo que es una red convolucional
\subsection{Redes Convolucionales}
\noindent
Las Redes neuronales convolucionales son  un tipo de redes neuronales artificiales  donde las \emph{neuronas}
corresponden a campos receptivos de una manera muy similar a las neuronas en la corteza visual primaria (V1) de un cerebro
biológico.  Este tipo de red es una variación de un perceptrón multicapa, sin embargo, debido a que su aplicación es realizada
en matrices bidimensionales, son muy efectivas para tareas de visión artificial, como en la clasificación y segmentación 
de imágenes, entre otras aplicaciones.

\subsection{SRCNN}
Como se menciona en \cite{freeman},
    \subsection{SRGAN}

El SISR (Single Image Super Resolution) es un problema inverso, \emph{it est}, que para una imagen de baja resolución puede haber muchas
imágenes diferentes de alta resolución que le correspondan, esto basado en la interpretación del método utilizado, ya que
el principio básico es añadir información para obtener imágenes de alta resolución.
Las CNN presentan un gran avance en la reconstrucción de imágenes de baja resolución a alta resolución,
sin embargo, debido al escalado de la imagen o el hecho de que la imagen que se busca mejorar presenta grandes
variaciones con respecto a las del dataset (\emph{Data Augmentation}) los resultados podrían ser no satisfactorios.


Una alternativa que propone un nuevo paradigma son las GAN´s (Generative Adversarial Networks), cuyo funcionamiento está basado
en la estimación de modelos generadores, como mencionan Goodfellow et al. \cite{GANs}, esto es 
posible gracias al entrenamiento simultáneo de dos modelos, uno \emph{generador (G)} que obtiene 
la distribución de la entrada para generar datos falsos y el otro \emph{discriminador (D)} el cual se encarga de estimar 
la probabilidad de que la muestra provenga del dataset de entrenamiento y discernir así entre estos datos y 
los del modelo \emph{ generador (G)}.

El término \emph{antagónicas} como se menciona en \cite{SRGAN_Tesis}, se refiere a la dinámica 
competitiva que se mantiene entre los dos modelos. Por un lado,
el generador tiene por objetivo crear nuevos datos que sean indistinguibles del
conjunto de entrenamiento, mientras que el discriminador debe poder ser capaz
de distinguir cuáles son los datos creados y los reales, siendo los últimos los que corresponden
 al conjunto de entrenamiento. Esto resulta en un proceso iterativo donde estos dos modelos
 se desafían uno a otro, logrando un ajuste de parámetros
 que logran producir datos que se parezcan con gran acierto a los reales.
 

\begin{figure}[H]
    \begin{center}
      \includegraphics[scale = 0.9]{modelo_gen_disc.png}
      \caption{Modelo Generador y Discriminador}
      \label{Alexis1}
    \end{center}
\end{figure}


El modelo SRGAN (Super Resolution Generative Adversarial Network) fue
propuesto en 2016 por un grupo de investigadores de la empresa Twitter \cite{SRGAN}.
La principal innovación de este modelo es su función de pérdida, llamada
función de pérdida perceptual, que permite mejorar el realismo de la imagen de
salida.

\subsubsection{Componentes.}

Profundizando un poco más en los componentes del algoritmo, en especifico, el uso de las \emph{GAN´s} para la obtención de Super-Resolución (\emph{SRGAN}), 
tenemos al discriminador el cual es una red neuronal convolucional que consta de muchas 
capas ocultas y una capa de salida, la principal diferencia aquí es que la capa de salida de las GAN puede tener solo dos salidas, 
a diferencia de las CNN, que pueden tener un número diferente de salidas con respecto a la cantidad de etiquetas en las que está entrenado.
La salida del discriminador puede ser 1 o 0 dependiendo de la función de activación que se aplique. Si la salida es 1, 
entonces los datos proporcionados son reales y si la salida es 0, entonces se refiere a ellos como datos falsos.

El discriminador está capacitado con los datos del dataset, con estos aprende a reconocer cómo se ven y qué características deben 
clasificarse como reales, formalmente, discrimina entre $\tilde{x}$, la muestra falsa, y $x$, 
los datos muestreados de la distribución real de datos $P_{datos}(x)$.




\begin{figure}[H]
    \begin{center}
      \includegraphics[scale = 0.4]{discriminador.png}
      \caption{Modelo Discriminador}
      \label{Alexis2}
    \end{center}
\end{figure}


Por el contrario, el generador es una red neuronal convolucional inversa, hace exactamente lo opuesto de lo que hace una CNN, ya que 
a estas se les da una imagen real como entrada y se espera una etiqueta clasificada como salida, 
pero en el generador, un vector de ruido aleatorio\emph{(z)} se da como señal de entrada 
y se espera una imagen falsa como salida, esta imagen deberá aproximarse a la real a
partir de una distribución $P(z)$ (en general una distribución Gaussiana) que
produce una muestra de datos falsos,$\tilde{x}$ es decir:

\begin{equation}
G(z) = \tilde{x}
\end{equation}

\begin{figure}[H]
    \begin{center}
      \includegraphics[scale = 0.6]{generador.png}
      \caption{Modelo Generador}
      \label{Alexis3}
    \end{center}
\end{figure}
    
\subsubsection{Funciones de perdida.}

Como parte del entrenamiento se necesitan parámetros que nos describan de manera correcta los resultados
que obtenemos de la predicción en nuestra red neuronal, para esto se hace el uso de funciones de perdida, estas funciones 
evalúan la desviación entre las predicciones realizadas por la red neuronal y los valores 
reales de las observaciones utilizadas durante el aprendizaje. A esta función se le conoce como perdida perceptual. Cuanto menor es el resultado de esta función, 
más eficiente es la red neuronal. Su minimización, es decir, reducir al mínimo la desviación entre el valor de la predicción y
el valor real para una observación dada, se hace ajustando los distintos pesos de la red neuronal.

En el caso de las redes adversarias, específicamente en \emph{SRGAN}, esta perdida es la suma de las perdidas de contenido $l_{X}^{SR}$ 
y las adversarias $10^{-3}l_{Gen}^{SR}$.


\begin{equation}
  l^{SR}=l_{X}^{SR} + 10^{-3}l_{Gen}^{SR}
\end{equation}


Como mencionan Goodfellow et al. \cite{GANs}, se tienen 
2 perdidas principales: La de contenido (\emph{content loss}), la cual se aproxima a una perdida perceptual mediante la perdida
de la red neuronal (\emph{VGG}),definimos entonces la perdida \emph{VGG} como la distancia euclidiana entre la representación de
características de una imagen reconstruida $G_{\theta G}(l^{LR})$ y la imagen de referencia o real $l^{HR}$.


\begin{equation}
  l_{VGG/i.j}^{SR}=\frac{1}{W_{i,j}H_{i,j}} \sum_{x=1}^{W_{i,j}}\sum_{y=1}^{H_{i,j}}(\phi_{i,j}(l^{HR})_{x,y}-\phi_{i,j} 
  (G_{\theta G}(l^{LR}))_{x,y})^{2}
\end{equation}

Por otra parte tenemos la perdida adversaria, esta se añade el componente generador de nuestro modelo GAN a la perdida perceptual,
esto promueve que nuestra red cree soluciones que engañen al discriminador. Las perdidas del generador $l_{Gen}^{SR}$ se definen
en base a las probabilidades del modelo discriminador $D_{\theta D}(G_{\theta G}(l^{LR}))$ sobre las muestras de entrenamiento
de la siguiente manera:

\begin{equation}
  l_{Gen}^{SR}=\sum_{n=1}^{N}-log \ D_{\theta D}(G_{\theta G}(l^{LR}))
\end{equation}


Donde $D_{\theta D}(G_{\theta G}(l^{LR}))$ es la probabilidad de que nuestra imagen reconstruida $G_{\theta G}(l^{LR})$ sea 
una imagen de alta resolución y la función logarítmica nos ayuda con el comportamiento del gradiente. 


    % Descripción de la SRCNN
\subsection{Red Neuronal Convolucional de Super Resolución: (SRCNN por sus siglas en inglés)}
Como se menciona en \cite{freeman}, tradicionalmente se utilizan 3 etapas para la super resolución de imagenes:
\begin{enumerate}
    \item Extracción de parches en baja resolución y representación en un vector de "alta" dimensión
    \item Mapeo entre vectores (parches de baja resolución) de baja resolución con vectores que representan parches de alta
    resolución
    \item Reconstrucción de imagén
\end{enumerate}
Estas 3 etapas generalmente se realizan por separado, es por ello que se propone el uso de una red convolucional en la cual se
lleven a cabo estas 3 operaciones de manera conjunta.\\
En \cite{SRCNN} se propone una red convolucional que consté de 3 capas, cada una correspondiente a las etapas que clasicamente
se siguen para el problema de super resolución, es decir, 1) Se diseña una primer capa convolucional con la cual se busca la
"extracción" de los parches de baja resolución, 2) una capa que estará conectada directamente con la salida de la primer capa
convolucional, y que será la encargada de realizar el "mapeo" entre los parches de baja resolución y los parches de alta
resolución y 3) una capa final encargada de realizar la reconstrucción de la imagen de alta resolución.

\begin{figure}[H]
    \label{fig:SRCNN_Arquitectura}
    \centering
    \includegraphics[scale = 0.6]{SRCNN_Arquitectura.png}
    \caption{Arquitectura de la SRCNN}
\end{figure}

Se busca generar un mapeo entre directo entre la imagen de baja resolución (Y) y la de alta resolución real (X).

\textbf{1. Extracción de parches}\\
Esta etapa comunmente se basa en la extracción de parches de baja resolución y representarlos por un conjunto de bases pre-entrenadas
tales como PCA,DCT,Haar,etc. Esto es equivalente a "convolucionar" la imagen por un conjunto de filtros, donde cada uno de estos
representaria las bases pre-entrenadas. En la formulación de la SRCNN, se envuelve la optimización de estas bases en la optimización
de la red. Formalmente, la primer capa se expresa como una operación $F_1$:
\begin{align}
    \label{eqn:SRCNN_FirstLayer}
    F_1(Y)=max(0,W_1*Y+B_1)
\end{align}
Donde $Y$ y $B_1$ representan los filtros y el \emph{bias} respectivamente y el operador "$*$" denota la operación de convolución.
Se utiliza la función de activación \textbf{ReLU} (ReLU,$max(0,x)$) en las respuestas de los filtros.\\

\textbf{2. Mapeo}\\
La primer capa extrae caracteristicas de $n_1$ dimensiones para cada parche. En la segunda operación se realiza un mapeo de cada
uno de estos vectores de dimensiones $n_1$ con vectores de dimensiones $n_2$. Esto es equivalente a aplicar $n_2$ filtros. Esta
descripción es válida para filtros de kernel $1\times1$, pero es facil generalizar para otros tamaños de \emph{kernel}, solo que
ahora el mapeo se realiza en parches de kernel $p\times p$. La operación de la segunda capa es:

\begin{align}
    \label{eqn:SRCNN_SecondLayer}
    F_2(Y)=max(0,W_2*F_1(Y)+B_2)
\end{align}

Cada uno de los vectores de salida de dimensiones $n_2$ es conceptualmente la reprsentación de un parche en alta resolución que
será utilizado para la reconstrucción.\\

\textbf{3. Reconstrucción}\\
En los métodos tradicionales, la predicción de traslapamiento de parches de alta resolución es frecuentemente promediada para
producir la imagen final. El promediado puede ser considerado como un filtro pre-definido sobre un conjunto de mapas de
características (Donde cada posición es la forma de vector "aplando" de un parche de alta resolución). Se define una capa
convolucional para producir la imagen final en alta resolución:

\begin{align}
    \label{eqn:SRCNN_ThirdLayer}
    F(Y)=W_3*F_2(Y)+B_3
\end{align}


    \section{Implementación}
    \noindent
Dado que el objetivo del proyecto es comparar tres diferentes métodos 
de súper resolución, en esta sección se presentarán las actividades
requeridas para la implementación de los métodos. Considerando desde los 
requisitos de software, hardware y datos de entrenamiento en caso sean 
necesarios. 

\subsection{Example-Based Super-Resolution}
\subsubsection{Construcción de diccionario}
\noindent
Para el primero de ellos, se retoma la literatura expuesta en \cite{freeman}
comenzando con la generación del diccionario donde se encuentran relacionados 
los parches de baja y alta resolución. Previo a esto, es necesario pre-procesar
el conjunto de imágenes para su posterior segmentación. 

De acuerdo a las indicaciones, se deben tener pares de imágenes en alta y baja 
resolución. Para esta implementación se ha considerado las primeras 14 imágenes
de \cite{MIRFLICKR} las cuales están en alta resolución. En la Figura \ref{fig:fr_dataset}
puede observarse que el conjunto de imágenes no guardan una relación en específico
por lo que los requisitos para aplicar el algoritmo son bastante flexibles.

\begin{figure}[H]
    \includegraphics[scale = 0.4]{ fr_dataset.png }
    \centering
    \caption{ \emph{Dataset} utilizado para construcción de diccionario}
    \label{fig:fr_dataset}
\end{figure}

En la Figura \ref{fig:fr_interpolacion} se presenta el procedimiento realizado para
construir el conjunto de imágenes de baja resolución mediante una reducción por el 
factor $\frac{1}{\Omega}$ y posterior escalado de $\Omega$ con el objetivo de 
perder información al escalar la imagen una vez realizada la compresión, forzando
la baja resolución. Para esta implementación se consideró un factor $\Omega = 4$. 

\begin{figure}[H]
    \includegraphics[scale = 0.6]{ fr_prepdic.png }
    \centering
    \caption{Preparación de base de entrenamiento mediante algoritmos de interpolación}
    \label{fig:fr_interpolacion}
\end{figure}

Como se presenta en la Figura \ref{fig:fr_interpolacion}, ambas imágenes 
rquieren ser filtradas (para reducir la cantidad de información innecesaria)
y normalizadas (con el objetivo de generalizar el algoritmo) para realizar el
proceso de segmentación. Para cada imagen filtrada $\text{img}?f$ se utilizó la función de\
\emph{OpenCV} \emph{GaussianBlur} considerando una desviación estándar $\sigma = 1$
con el fin de restar la imagen desenfocada (frecuencias medias y bajas) a la imagen 
original para dejar sólo las frecuencias altas (detalles) tal como se presenta
en la expresión \eqref{eqn:filtroPH}.

\begin{align}
    \label{eqn:filtroPH}
    \text{img}_f = \text{img} - \text{\emph{cv2.GaussianBlur(img, }(0,0),}\, \sigma)
\end{align}

Para obtener la imagen normalizada $\text{img}_n$ se consideró la expresión \eqref{eqn:normalizado}.

\begin{align}
    \label{eqn:normalizado}
    \text{img}_n = \frac{\text{img}_f-\text{min}(\text{img}_f)}{\text{max}(\text{img}_f)-\text{min}(\text{img}_f)}\cdot 255
\end{align}

De esta manera, el \emph{segmentador} puede realizar un barrido bidimensional (de acuerdo a 
la Figura \ref{fig:fr_dic}) en cada imagen 
con el objetivo de recolectar los parches en alta y baja resolución con 
dimensiones de $5x5$ y $7x7$ respectivamente.

\begin{figure}[H]
    \includegraphics[scale = 1.2]{ fr_segmentado.png }
    \centering
    \caption{ Segmentación y almacenamiento de parches en diccionario }
    \label{fig:fr_segmentador}
\end{figure}

Posteriormente, el parche de baja resolución es reordenado para formar un 
vector fila en $\mathbb{R}^{1\times147}$ para concatenarse con el vector de superposición
(en $\mathbb{R}^{1\times30})$ que considera la primera fila y primera columna del
parche de alta resolución. A dicha concatenación se le nombrará como 
\emph{vector de búsqueda}. En la Figura \ref{fig:fr_segmentador} puede observarse
al conjunto de vectores de búsqueda que serán guardados en el \emph{diccionario}
junto con los parches de alta resolución. 

El proceso anterior es realizado para cada imagen y dado que cada una es de 
dimensiones diferentes, el número de parches será distinto. Cabe 
destacar que el almacenamiento de los 154,125 vectores realizó
mediante el paquete \emph{h5py} en el archivo \emph{diccionario.h5},
ya que es recomendable para su implementación en algoritmos de búsqueda. 

\subsubsection{Algoritmo de búsqueda}
\noindent
Una vez construido el \emph{diccionario}, es necesario establecer el algoritmo de 
búsqueda para que cuando se realice la segmentación de la imagen a mejorar 
su resolución se busquen los parches de alta resolución de acuerdo al 
\emph{vector de búsqueda}.

Para esto, \cite{freeman} propone el uso de algoritmos de búsqueda como 
\emph{el vecino más cercano} para encontrar al vector más próximo a la solicitud
y con ello asociar al parche de alta resolución correspondiente. 

Utilizando la función \emph{NearestNeighbors} del paquete \emph{sklearn.neighbors} \cite{scikit-learn},
la implementación resulta sencilla una vez ordenados todos los vectores en el 
diccionario. 

Para verificar que el algoritmo funciona adecuadamente, se ha propuesto un 
parche externo de baja resolución (\emph{solicitud}) para 
que encuentre al parche de baja resolución más cercano que esté dentro del
\emph{diccionario} y asocie con el parche de alta resolución. En la Figura
\ref{fig:fr_vecinos} puede observarse que el primer pixel tiene un color 
diferente y aun así realiza la aproximación adecuadamente en 8.9 $s$
utilizando el algoritmo \emph{Ball Tree}.  

\begin{figure}[H]
    \includegraphics[scale = 0.6]{ fr_vecinos.png }
    \centering
    \caption{ Ejemplo de búsqueda de parches en diccionario}
    \label{fig:fr_vecinos}
\end{figure}

Respecto a la complejidad temporal de cada algoritmo, se presenta en 
la Tabla \ref{fig:fr_vecinos} otros algoritmos 
que de acuerdo a \cite{NN_search} cuenta la función utilizada.

\begin{table}[H]
    \caption{Tiempos de algoritmo de búsqueda}
    \label{tb:tiempos_snn}
    \centering
    \begin{tabular}{|c|c|}
    \hline
    \textbf{Algoritmo}    & \textbf{Tiempo} {[}s{]} \\ \hline
    Auto         & 9.8            \\ \hline
    Fuerza Bruta & 9.5            \\ \hline
    KD Tree      & 9.5            \\ \hline
    Ball Tree    & 8.9            \\ \hline
    \end{tabular}
\end{table}

Note que para fines de eficiencia considerando $D$ como dimensiones 
y $N$ como número de elementos, resulta más conviente el uso del algoritmo
\emph{Ball Tree} dado que complejidad es $\mathcal{O}[D\,\text{log}|N| ]$,
mientras que para el algoritmo de \emph{fuerza bruta} es
$\mathcal{O}[D\,\text{log}|N| ]$. Por otro lado, el algoritmo \emph{KD Tree}
tienen una complejidad $\mathcal{O}[D\,\text{log}|N| ]$ cuando $D$ es menor a 20
dimensiones. Sin embargo, si $D$ aumenta, entonces su complejidad resulta
$\mathcal{O}[D\,N]$. Por lo mismo, \emph{Ball Tree} es el algoritmo más rápido
para esta aplicación dados los resultados de la Tabla \ref{tb:tiempos_snn}. 

\subsubsection{Algoritmo de predicción}
\noindent
Teniendo ya el diccionario y el algoritmo de búsqueda, es posible aplicar el
\emph{algoritmo de un paso} propuesto por \cite{freeman} para mejorar la 
resolución de la imagen de entrada. Para ello nos basamos en el diagrama
presentado en la Figura \ref{fig:fr_algoritmo} realizando cada uno de los 
pasos mediante un paquete desarrollado por nosotros en Python. 

En la Figura \ref{fig:fr_proceso} puede observarse el proceso realizado para mejorar 
la resolución de la imagen ejemplo. Para este caso se ha considerado un escalado
con factor  $\Omega=2$ y factor de superposición de $\alpha=0.05$.

\begin{figure}[H]
    \includegraphics[scale = 0.5]{ fr_proceso.png }
    \centering
    \caption{ Algoritmo de Súper Resolución}
    \label{fig:fr_proceso}
\end{figure}

De igual manera se han realizado variaciones en el factor de superposición
con el objetivo de encontrar el óptimo. En la Figura \ref{fig:fr_alphas}
se presentan algunos de los valores utilizados. 

\begin{figure}[H]
    \includegraphics[scale = 0.4]{ fr_alphas.png }
    \centering
    \caption{ Variación en parámetro de superposición $\alpha$ }
    \label{fig:fr_alphas}
\end{figure}

Observe que el parámetro de superposición permite enfatizar los detalles de
la imagen de altas frecuencias e indirectamente los detalles de la
imagen con mejor resolución. 


    \subsection{SRCNN}
Partimo del método descrito en \cite{SRCNN}, en el cual se menciona una red que cuente con al menos 3 capas convolucionales, de
manera que la primer capa se encargue de extraer los parches de baja resolución, la(s) capa(s) intermedias realizan el mapeo entre
los parches de baja resolución y los de alta resolución, y la última capa realizará la reconstrucción de la imagen de alta
resolución. Para ello hicimos uso de la libreria \textbf{keras} y de \textbf{tensorflow}.\\
\subsubsection{Entrenamiento de la red}
Para poder tener una SRCNN funcional, tuvimos que entrenar nuestro modelo a fin de lograr obtener los resultados que esperabamos.
Para este proceso de entrenamiento es necesario que en nuestro \emph{dataset} tengamos la imagen en alta resolución y su
correspondiente par de baja resolución. Para obtener las imagenes en baja resolución, se puede seguir el siguiente procedimiento:
\begin{enumerate}
    \item Reducir la escala de la imagen original (Alt resolución), puede ser en un factor de 2 o el que se desee.
    \item Reescalar la imagen que acabamos de reducir, dependiendo del factor de reducción de escala que hayamos utilizados, de
    manera que tengamos una imagen del mismo tamaño que la imagen de alta resolución. El resultado será una imagen que se verá
    borrosa, es decir, habremos reducido su resolución.
\end{enumerate}
Debido a esta forma de generar las imagenes de baja resolución, fue necesario tener un módelo SRCNN para cada factor de escalamiento
de imagen. Para nuestro caso se realizaron entrenamientos para 2 factores de escalamiento: \textbf{1)} Factor de 2 y \textbf{2)}
factor de 4.\\
Tambien fue necesario tomar en cuenta el espacio de color en el cual se realizará el entrenamiento, ya que de este depende la
\emph{forma} de la imagen de entrada de la red neuronal. Para nuestro casó realizamos modelos SRCNN para los espacios de color
\textbf{RGB} y \textbf{$YC_rC_b$}.\\

%%%%%%%%%%%%%%%%%%%%%%%%%%%%%%%%%% Texto que se podria utilizar en la sección de "discusiones" %%%%%%%%%%%%%%%%%%%%%%%%%%%%%%%%%%
\begin{comment}
Para el caso del espacio de color $YC_rC_b$ realizamos el entrenamiento únicamente en el
canal $Y$, ya que es en este canal en donde se concentra la mayor parte de la información de la imagen referente a los detalles y
texturas (Si vemos este canal por separado, sería como ver la imagen en escala de grises). Para el caso del espacio de color $RGB$
el entrenamiento fue realizado sobre los 3 canales, ya que en este espacio de color, los detalles y texturas de la imagen se
distribuyen entre los 3 canales.\\
\end{comment}

%%%%%%%%%%%%%%%%%%%%%%%%%%%%%%%%%%%%%%%%%%%%%%%%%%%%%%%%%%%%%%%%%%%%%%%%%%%%%%%%%%%%%%%%%%%%%%%%%%%%%%%%%%%%%%%%%%%%%%%%%%%%%%%%%

    \subsection{Implementación SRGAN}

\subsubsection{Entrenamiento}

    \section{Resultados}
    \noindent
Una vez presentados los fundamentos teóricos y las diferentes actividades
para la implementación de cada uno de los métodos, se presentará 
un ejemplo de aplicación. 

Para ello, se propone un \emph{zoom} digital para imágenes panorámicas principalmente
dadas coordenadas de interés. De este modo, en las Figuras \ref{fig:saltillo},
\ref{fig:villahermosa} y \ref{fig:comitan} se presentan fotografías de las 
ciudades de Saltillo, Villahermosa y Comitán respectivamente. Donde para cada caso
se ha propuesto un área de interés de $32x32$ pixeles representada con
un recuadro verde. 

\begin{figure}[H]
    \includegraphics[scale = 0.25]{ ResultadoSaltillo.png}
    \centering
    \caption{Zoom digital en fotografía de Saltillo, Coahuila}
    \label{fig:saltillo}
\end{figure}

Del mismo modo, en cada una de las figuras se encuentra explícitamente
el área de interés o \emph{de zoom}, interpolación bicúbica con el objetivo
de representar y comparar los efectos comentados en la sección de \emph{Antecedentes}.
Así como los tres métodos considerados en este documento para su comparativo.

\begin{figure}[H]
    \includegraphics[scale = 0.25]{ ResultadoVilla.png}
    \centering
    \caption{Zoom digital en fotografía de Villahermosa, Tabasco}
    \label{fig:villahermosa}
\end{figure}

Observe que el comportamiento para cada uno de los métodos es el esperado de
acuerdo a las ventajas y desventajas mencionadas aun cuando el factor de escalado 
es de cuatro veces. 

\begin{figure}[H]
    \includegraphics[scale = 0.25]{ ResultadoComitan.png}
    \centering
    \caption{Zoom digital en fotografía de Comitán, Chiapas}
    \label{fig:comitan}
\end{figure}


    \section{Conclusiones}
    \begin{frame}{Conclusiones}
    \begin{block}{}
        \begin{itemize}
            \item La predicción de frecuencias altas resultó útil cuando se 
            requiere mejorar la resolución de una imagen.
            \pause
            \item \emph{SRCNN} y \emph{SRGAN} son más robustas y rápidas en la predicción 
            de la imagen para mejorar su resolución. 
            \pause
            \item El algoritmo de \emph{Freeman et al} tiene una muy buena aproximación
            y su 'entrenamiento' no es tan tardado como los métodos con redes neuronales. 
            \pause
            \item El comportamiento de los tres métodos fue el esperado y representan
            una evolución en tareas de \emph{Súper Resolución}. 
        \end{itemize}
    \end{block}
\end{frame}


    \section{Discusión}


    \Urlmuskip=0mu plus 1mu\relax
    \bibliographystyle{IEEEtran}
    \bibliography{bibliografia}

\end{document}